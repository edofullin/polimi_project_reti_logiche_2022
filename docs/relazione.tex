\documentclass[12pt, a4paper]{article}
\usepackage[outputdir=out]{minted}
\usepackage[utf8]{inputenc}
\usepackage{geometry}
\usepackage[italian]{babel}
\usepackage{graphicx}
\usepackage[belowskip=-15pt,aboveskip=0pt]{caption}
\usepackage{enumitem}

\setlist{nosep}

\graphicspath{{images/}}

\geometry {   
    top=20mm,
    bmargin=20mm
}
\begin{document}

\begin{titlepage}
    \centering

    \vspace{0.5cm} {
        \large Politecnico di Milano\\
    }

    \vspace{5cm} {
        \huge {
            Progetto di Reti Logiche 2021/2022\\
        }
        \vspace{0.5cm}
        \large {Scaglione Prof. Salice}
    }

    \vspace{2cm} {
        \large
        Edoardo Fullin ( Codice Persona \input{codice_persona.tex})
    }

    \vspace*{\fill}
    \today

\end{titlepage}

\tableofcontents

\pagebreak

\section{Introduzione}

\subsection{Descrizione del progetto}

Lo scopo del progetto è il design, tramite linguaggio VHDL di un modulo hardware sintetizzabile per FPGA
in grado di applicare ad una sequenza di parole in input un codice convoluzionale $1/2$, che 
può essere implementato attraverso la macchina a stati finiti in figura.


\begin{figure}[!ht]
    \centering
    \includegraphics[scale=0.7]{convoluzionatore_1_2_fsm.png}
    \caption{FSM convoluzionatore $1/2$}
    \label{fig:fsm_conv}
\end{figure}

Il modulo deve quindi restituire in output (scrivendo in memoria) le uscite della macchina a stati finiti,
bit per bit. % dire meglio

\subsection{Esempio}

La FSM prende in input la parola \verb+11001011+, che deve essere serializzata come
1 al tempo 0, 1 al tempo 1, 0 al tempo 2 e così via.
La macchina quindi, partendo dallo stato di reset 00 andrà nello stato 10 alla lettura
del primo 1 producendo in output 11, passerà poi nello stato 3 producendo 10, nello stato 01
producendo 10 e nello stato 00 producendo 11.
Il primo byte di output sarà quindi 11100111, che dovrà essere scritto in memoria.
Si nota quindi facilmente che la lunghezza della sequenza di output è esattamente doppia di quella 
dell'input.

\pagebreak

\subsection{Specifiche del modulo da realizzare}

Il modulo da descrivere è sincrono sincronizzato su un clock 'globale', non include la memoria su cui viene effettuato input/output e
deve implementare la seguente interfaccia verso l'esterno:

\begin{minted}[]{vhdl}
    
    entity project_reti_logiche is
        port (
            i_clk : in std_logic;
            i_rst : in std_logic;
            i_start : in std_logic;
            i_data : in std_logic_vector(7 downto 0);
            o_address : out std_logic_vector(15 downto 0);
            o_done : out std_logic;
            o_en : out std_logic;
            o_we : out std_logic;
            o_data : out std_logic_vector (7 downto 0)
        );
    end project_reti_logiche;

\end{minted}

In particolare:

\begin{itemize}
    \item \texttt{i\_clk} è il segnale di clock, generato dall'esterno
    \item \texttt{i\_rst} è il segnale di reset
    \item \texttt{i\_start} è il segnale di inizio sequenza
    \item \texttt{i\_data} è il byte in arrivo dalla memoria RAM
    \item \texttt{o\_address} è l'indizzo di memoria da cui si vuole leggere/scrivere
    \item \texttt{o\_done} è il segnale di fine computazione
    \item \texttt{o\_en} è il segnale di enable per la memoria RAM
    \item \texttt{o\_we} è il segnale di scrittura per la memoria RAM
    \item \texttt{o\_data} è il byte da scrivere in memoria RAM
\end{itemize}

\pagebreak

\section{Architettura}

\subsection{Macchina a stati}

Il modulo è composto da una macchina a 14 stati il cui compito è quello di generare
i segnali di controllo in grado di attivare o disattivare le parti del datapath atte a
svolgere la funziona richiesta in quello specifico punto dell'esecuzione.\\

La macchina a stati è descritta come in figura, quando non è presente una label sulle transizioni
si intende che la transizione è incondizionata oppure, in presenza di altre transizioni
in uscita dallo stesso stato, quando la condizione associata all'altra transizione è falsa.
E' inoltre presente, ma non riportata nel diagramma, una transizione da tutti gli stati 
verso \texttt{SINIT} quando \texttt{i\_rst = 1}

\begin{figure}[!h]
    \centering
    \includegraphics[scale=0.3]{fsm_controllo.png}
    \label{fig:ctrl_fsm}
    \caption{FSM di controllo}
    
\end{figure}

\subsubsection{INIT}
Stato iniziale in cui si trova la macchina prima dell'avvio dell'esecuzione.\\
I segnali di controllo sono tutti disattivati

\subsubsection{SINIT}
Stato in cui si trova la macchina prima dell'avvio di una nuova sequenza\\
I segnali di controllo sono tutti disattivati

\subsubsection{RST}
Reset di tutte le memoria interne al datapath per riportare rendere il datapath pronto per
una nuova sequenza.\\
I segnali di controllo attivi sono:
\begin{itemize}
    \item \texttt{sm\_rst} che resetta stato corrente e uscita corrente del convoluzionatore
    \item \texttt{curr\_mux} ad 11 che resetta a la posizione corrente
    \item \texttt{outbuff\_rst} che resetta il buffer di output
    \item \texttt{outbuff\_load} che è necessario per resettare il bufferi di output
\end{itemize}

\subsubsection{RNB}
Lettura del numero di byte dalla RAM ad indirizzo 0.\\
I segnali di controllo attivi sono:
\begin{itemize}
    \item \texttt{nbytes\_load} che attiva il registro che salva il numero di byte
    \item \texttt{curr\_mux} ad 10 che fa avanzare la posizione corrente al prossimo byte
\end{itemize}

\subsubsection{SWNB}
Verifica lo stato corrente ed il termine della sequenza.\\
I segnali di controllo sono tutti disattivati

\subsubsection{RB}
Lettura del byte corrente e memorizzazione.\\
I segnali di controllo attivi sono:
\begin{itemize}
    \item \texttt{sr\_byte\_load} che attiva la lettura del registro che contiene il byte
\end{itemize}

\subsubsection{SWB}
Stato di stallo, in attesa che il byte veng letto con successo.\\
I segnali di controllo attivi sono:
\begin{itemize}
    \item \texttt{sr\_byte\_load} che attiva la lettura del registro che contiene il byte
\end{itemize}

\subsubsection{SM}
Il bit corrente è ora presente all'ingresso del convoluzionatore, che viene abilitato e viene
generato l'output relativo a quel bit.\\
I segnali di controllo attivi sono:
\begin{itemize}
    \item \texttt{sr\_ena} che attiva il registro serializzatore
    \item \texttt{sm\_ena} che attiva il convoluzionatore
\end{itemize}

\subsubsection{W0 e W1}
I bit di output vengono scritti nell'output buffer.\\
Al termine della scrittura si verifica il segnale \texttt{outbuff\_full} per decidere
se è necessario scrivere il bufferi di output in memoria.
I segnali di controllo sono:
\begin{itemize}
    \item \texttt{sm\_w\_sel} rispettivamente a 0 in W0 e 1 in W1 che indica se viene 
                              scritto il bit 0 alla posizione corrente o il bit 1 alla posizione corrente + 1
    \item \texttt{outbuff\_load} che abilita la scrittura nel bufferi di output
\end{itemize}

\subsubsection{WB}
Scrittura in memoria del buffer di output.\\
I segnali di controllo attivi sono:
\begin{itemize}
    \item \texttt{writesel} che attiva il la scrittura in memoria e genera l'indirizzo corretto
\end{itemize}

\subsubsection{RSTOB}
Reset del buffer di output.\\
I segnali di controllo attivi sono:
\begin{itemize}
    \item \texttt{outbuff\_rst} che resetta il buffer di output
    \item \texttt{outbuff\_load} che è necessario per resettare il bufferi di output
\end{itemize}

\subsubsection{FIN}
Il singolo bit è stato processato, viene avanzato il contatore.\\
Si verifica il segnale \texttt{cbit\_end} per decidere se è necessario passare alla lettura del
prossimo byte oppure del prossimo bit.
I segnali di controllo attivi sono:
\begin{itemize}
    \item \texttt{curr\_mux} a 01 che avanza il contatore di una posizione
\end{itemize}

\subsubsection{DONE}
La sequenza è terminata, in attesa di i\_start o i\_rst per iniziare una nuova sequenza.\\
I segnali di controllo attivi sono:
\begin{itemize}
    \item \texttt{o\_done} a 1 segnala il termine della computazione
\end{itemize}

\subsection{Datapath}

Il datapath è il componente che si occupa della gestione del flusso dati, incluso l'input e l'output
da e verso la memoria RAM.
Nella attuale implementazione è stata usata una solo entity VHDL, che può essere vista come l'insieme
di più moduli collegati tra loro come da schema in figura.
I componenti del datapath sono pilotati dai segnali di controllo generati dalla macchina a stati che
controlla il flusso di esecuzione.

Uno schema di massima è rappresentato in figura, lo schema completo è a pagina xxx


\begin{figure}[h!]
    \includegraphics[scale=0.32]{datapath.png}
\end{figure}

\subsubsection{Input Buffer}

Il buffer di input è realizzato con un registro che, in presenza del segnale 
\texttt{sr\_byte\_load} memorizza il dato attualmente presente sul bus \texttt{i\_data}.
Il registro ha inoltre un parametro aggiuntivo \texttt{i\_sel} di lunghezza 3 bit (0-7)
che permette di specificare quale bit si vuole in ouput.

\subsubsection{Contatore}

Il contatore è un contatore da 11 bit che conta il numero del bit corrente,
in questo modo gli ultimi 3 bit (0-7) rappresentano il numero del bit corrente
mentre i primi 8 bit rappresentano il numero (e quindi l'indirizzo) del byte corrente.

\subsubsection{Convoluzionatore}

Il convoluzionatore (\texttt{state\_machine.vhd}) è una macchina a stati finiti dotata di enable e reset che
realizza il convoluzionatore 1/2 come da specifica.

\subsubsection{Calcolatore Indirizzo di Output}

Il calcolatore dell'indirizzo di output è una ALU completamente combinatorica che, a partire
dal numero del bit corrente (in arrivo dal contatore) permette il calcolo dell'indirizzo di output.
Visto che il convoluzionatore per ogni bit di input restituisce sempre 2 bit l'indirizzo di output
sarà sempre $2 * curr\_pos + sm\_w\_sel$ dove \texttt{curr\_pos} è l'indirizzo corrente in arrivo dal
contatore mentre \texttt{sm\_w\_sel} è un segnale di controllo generato dalla macchina a stati che è attivo
(ad 1) se e solo se si vuole scrivere il secondo byte di output dal convoluzionatore.
Nel caso in cui la scrittura in memoria sia abilitata, viene anche sommata la costante \texttt{998}.
\\
\\
A causa del fatto che l'indirizzo di output deve essere (almeno) un bit più lungo di quello
di input si è deciso di stritturare tutto il componente per lavorare direttamente su indirizzi da 16 bit 
(che è la dimensione richiesta dalla memoria), è quindi richiesto il padding sull'indirizzo corrente
restituito dal contatore.


\subsubsection{Buffer di Output}

Il buffer di output è un componente sequenziale/combinarico che memorizza temporaneamente il risultato della
computazione, in attesa della scrittura in memoria. 
Il componente si rende necessario a causa dell'indirizzamento al byte della memoria.

Il buffer, quando abilitato alla scrittura, prende uno dei due bit in arrivo dal convoluzionatore tramite un mux 
pilotato dal segnale \texttt{sm\_w\_sel}, lo shifta a sinistra del numero di posizioni in arrivo
dal calcolatore dell'indirizzo di output e lo scrive nella posizione corretta 
mettendolo in bitwise OR con il valore attuale.
\pagebreak
\section{Simulazione}

\section{Sintesi}

Il componente è correttamente sintetizzabile e con la versione corrente di Vivado è implementabile su hardware 
XXXX usando XXX Lookup Tables e XXX Flip-Flop.
\\
\noindent In particolare, i XXXX flip flop sono così ditribuiti:
\begin{itemize}[itemsep=4pt, topsep=4pt]
    \item 4 bit per la macchina a stati di controllo
    \item 2 bit per lo stato corrente del convoluzionatore
    \item 2 bit per l'uscita corrente del convoluzionatore
    \item 8 bit per il buffer di output
    \item 11 bit per la posizione corrente
    \item 8 bit per il byte corrente
    \item 8 bit per il numero di bytes da leggere
    \item XXX \\
\end{itemize}

\noindent La sintesi non genera latch.

%% TIMING DI SINTESI?

\section{Simulazione e casi di test}

Il componente è stato simulato su una varietà di casi di test, che comprendono anche
gli edge-case più comuni.
Viene riportato l'andamento della simulazione per una selezione ci casi di test.

Sono stati effettuati i seguenti casi di test, dei quali per alcuni viene riportato parte dell'esecuzione: %% TODO

\begin{itemize}[itemsep=4pt, topsep=4pt]
    \item test con reset durante l'esecuzione
    \item test con sequenza di lunghezza zero
    \item test con sequenza di lunghezza massima (255)
    \item test con cambio della sequenza al termine dell'esecuzione
    \item test con sequenze generate casualmente
\end{itemize}

\subsubsection{Test con sequenza vuota}

Nel caso la sequenza sia vuota (RAM[0] = 0), dopo la scrittura del numero di bytes nell'apposito registro,
il datapath solleverà quindi immediatamente il segnale \texttt{seq\_end} che porterà la macchina a stati
nello stato di DONE senza mai scrivere in memoria.

\begin{minted}{vhdl}
    seq_end <= '1' when unsigned(curr_pos(10 downto 3)) > nbytes or 
                        curr_pos_overflow = '1' else '0';
\end{minted}

%% TODO

\subsubsection{Test con sequenza di lunghezza massima}

Il test con sequenza si lunghezza massima è stato quello con il maggior numero di problemi.
Questo perchè dopo aver processato l'ultimo byte il registro \texttt{curr\_pos} va in overflow
nello stato \texttt{FIN} tornando a zero.
\noindent Questo comporta che quando la macchina a stati torna nello stato \texttt{SWNB} il registro \texttt{curr\_pos} 
vale zero, che è minore di \texttt{nbytes}, quindi l'esecuzione riprende dall'inizio.

\noindent Per risolvere il problema è stato quindi introdotto un segnale \texttt{curr\_pos\_overflow} che viene alzato in maniera sequenziale
quando il registro \texttt{curr\_pos} passa da 255 a 0 ed il segnale che pilota il contatore è in posizione "+1" (01)

\noindent Il segnale di fine sequenza viene alzato (anche) quando questo nuovo segnale è ad 1.

\begin{minted}{vhdl}
    seq_end <= '1' when unsigned(curr_pos(10 downto 3)) > nbytes or 
                        curr_pos_overflow = '1' else '0';
\end{minted}

%% TODO

\subsubsection{Reset}

Quando viene alzato il segnale di reset la FSM di controllo passa sempre nello stato SINIT, ricominciando l'esecuzione.\\
\noindent Quando poi viene alzato il segnale di start la FSM di controllo passa nello stato \texttt{RST}
che provvedere al reset di tutti i componenti interni del datapath e del convoluzionatore.

\begin{minted}{vhdl}
    if i_rst = '1' then
        curr_state <= S_SINIT;
\end{minted}

\subsubsection{Test con sequenze multiple in fila}

Alla fine della sequenza la FSM di controllo alza il segnale \texttt{o\_done}, nel caso in cui \texttt{i\_start} venga abbassato
la macchina si porta nello stato SINIT e si prepara ad accettare una nuova sequenza.

%% TODO

\section{Scelte progettuali}

Si è scelto di scrivere il codice VHDL "a più basso livello" possibile in modo
che la architettura risultante sia il più possibile simile a quella progettata.
Si sarebbe potuto, in alternativa, usare più strutture \texttt{case} ed \texttt{if - elsif} 
al posto di usare MUX e segnali aggiuntivi.








\end{document}